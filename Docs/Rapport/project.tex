\chapter{Project implementation}

\section{Presentation of the Map-Matching method}

The objective is to match GPS points to roads by constructing a Hidden Markov Model with roads as states and GPS points as observations. Once the Hidden Markov Model is built, a dynamic programming technique, called Viterbi algorithm, is applied to find the most probable sequence of roads.

The paper (ref) presents how the matrices of the HMM are built and the robustness of the algorithm with different time steps for the GPS track.

The emission matrix $E$ reports the probabilistic relation between a point and the surrounding roads (up to 200m). 
The closer the point $p$ is to a road $r$ the higher the probability $E(p,r)$ is.
	
The transition matrix $M$ reports the probability of going to road $r_j$ from a road $r_i$. In the HMM theory, the probability of transitioning is independent of the observations.

In the paper, this probability depends on the distance between two consecutive points (straight distance) and the distance between the projection of those points on the roads.

A few constants are introduced (but their real value not given in the paper) for the probability computations. For instance, $\sigma_z$ represents the standard deviation of GPS measurements and is part of the computation of the emission matrix.  

\section{Input data}

The map matching algorithms works with GPS points and roads expressed as coordinates (x,y) in the metric system. 
The data presented on the paper is freely available in the latitude and longitude system including other informations such as one-ways and average speed (for roads).

This input data, as well as data from Paris, has been processed through \textbf{QGIS} to produce ready to use datasets :
\begin{itemize}
	\item a full dataset (road network and GPS track) from the paper representing a trip of more than 7500 points in Seattle (2500+ roads);
	\item a small dataset from the same inputs (231 GPS points);
	\item datasets of trips in Paris.
\end{itemize}

For each of these datasets, we have, at least :
\begin{description}
	\item[Tracks] x,y in meters and a timestamp;
	\item[Network] a set of coordinates describing the extremities of each segment of a road.
\end{description}

\section{Implementation hazards}

Once the input data where available in the adequate form (sufficient information expressed in meters), we could focus on the difficulties of the algorithms and the implementations options. 

A basic UML class diagram was produced in order to initiate the agile process. The principle being that each sprint would lead to a prototype expanding the previous sprint's prototype, hence adding more classes and functionalities to the software.

During the development phase, Doxygen was used to generate updated UML class diagrams.

<uml pic>

We have simplified the computation of both the emission matrix and the transition matrix. This was a side effect of our choice to not use a spatial library.

In the end, we were able to get a similar results as the one shown on the paper. Our probabilities don't follow the same distribution and its shows after a long strand of GPS point corrections : as each probability depends on all the previous steps, its value can only go down. Using \textit{double} instead of \textit{float} only delay the moments where the values of all the probabilities for one timestep are almost null.  


\section{Implementation details}

From the project description, we were advised to use C++, OpenGL through a library, QGIS, Globe, and Unit Tests.

A previously mentioned, QGIS has been used to transform input data, and more generally, to visualize tracks and road networks. 

During the first sprint, we focused on getting a grip on C++ QGIS plugins and Globe. Our conclusion was that not only it is very unlikely that we would succeed in compiling the whole QGIS project on Debian\footnote{QGIS includes Globe, so by setting all the required libraries with their correct versions for QGIS, we should have been able to compile Globe.}, but the development of a code as a plug in would drastically slow down our progress. As a stand alone application, it was way easier to detect any bugs and focus on our own code. The drawback is that none of the functionalities of QGIS (like computing distance, selecting roads in an area) where available to us. 

\subsection{Unit tests}

Our development has been done with Qt Creator and a limited use of Qt libraries. Qt offers a Unit Test library (Qt Test). A quick research showed that a large amount of C++ projects uses GoogleTest. We think that it is a better investment to use a well established standard. 
Google Test is relatively easy to setup and use (a short documentation was written to get everyone up to speed). We only tackled a tiny part of the possibilities of GTest as we focused on checking basic functionalities of our code (related to GSP tracks and road segments), but we are confident that we can go deeper on a future project. 

\subsection{Visualization and spatialization}

A big side effect of our decision to not use QGIS was the problem of visualization and more specifically of spatialization.

Qt is a very rich development framework. Among all the libraries, is \textit{GraphicsScene} and \textit{GraphicsView}. Whith those two classes, we could easily display, zoom and navigate on a road network, and, more importantly, take opportunity of its embedded partitioning to quickly select a set of neighboring roads.

\subsection{Other concepts}

In order to have a responsive UI (when loading files and during the visualization of the track correction), we have separated all the GUI classes and the "logic" classes in two threads, a feat easily done in Qt thanks to its Qt thread library.

Communication between threads is performed with the slot/event system provided by Qt.

GUI classes helped to create a user friendly interface. 



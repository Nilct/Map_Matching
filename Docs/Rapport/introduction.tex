\chapter{Introduction}

Since the arrival of smart phones equipped with GPS sensors, the uses of GPS have diversified and become more common.
The navigation's help is one of the applications of localization systems. It makes possible to optimize both professional and personal travels.

\section{Problem and resolution}

But if the uses have increased, the accuracy is not better, caused by the GPS technology. It therefore remains necessary to "make the position more reliable" for certain needs, such as navigation.
It's possible to determine the more likely road with map matching solution.

\subsection{Aim}

The project's aim is to make a tool that move GPS tracks onto the network, which is all the roads.

Cela inclut :
\begin{itemize}
\item{} un processus de map matching pour un lot de traces GPS
\item{} un processus de map matching pour une acquisition en temps réel
\item{} une interface graphique pour visualiser les données 3D (traces, réseau routier, fond de carte, \dots)
\end{itemize}

\subsection{Organization}

To work on the project, we have about 180 hours between October 3th and December 6th, the presentation's day.

The team is composed of seven students. The members are :\\
as scrum master : Romain \textsc{Mazière}, \\
as product owners :
\begin{itemize}
	\item Rudolf \textsc{Millet}
	\item Camille \textsc{Parisel}
\end{itemize}
as developers :
\begin{itemize}
	\item Hanane \textsc{Derbouz}
	\item Julie \textsc{Marcuzzi}
	\item Romain \textsc{Mazière}
	\item Loïc \textsc{Messal}
	\item Rudolf \textsc{Millet}
	\item Camille \textsc{Parisel}
	\item Mamady \textsc{Samassa}
\end{itemize}
as testers :
\begin{itemize}
	\item Camille \textsc{Parisel}
	\item Mamady \textsc{Samassa}
\end{itemize}
\chapter{Introduction}
La localisation exacte des biens et des personnes est un enjeu fondamental du XXI\ieme siècle. L'aide à la navigation est l'une des applications de la localisation. Elle a permis d'optimiser des trajets aussi bien commerciaux que personnels.
\section{Présentation du projet}
\subsection{Contexte}

\subsection{Objectif}
L'objectif de ce projet est de développer un outil de recalage de traces GPS. Cela inclut :
\begin{itemize}
\item{} un processus de map matching pour un lot de traces GPS
\item{} un processus de map matching pour une acquisition en temps réel
\item{} une interface graphique pour visualiser les données 3D (traces, réseau routier, fond de carte, \dots)
\end{itemize}
\subsection{Ressources}

\subsection{\'Equipe}
\subsection{Outils utilisés}
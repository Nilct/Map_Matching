\chapter{Introduction}
La localisation exacte des biens et des personnes est un enjeu fondamental du XXI\ieme siècle. L'aide à la navigation est l'une des applications de la localisation. Elle a permis d'optimiser des trajets aussi bien commerciaux que personnels.
\section{Présentation du projet}
\subsection{Contexte}

\subsection{Objectif}
L'objectif de ce projet est de développer un outil de recalage de traces GPS (map matching) sur le réseau routier. Cela inclut :
\begin{itemize}
\item{} un processus de map matching pour un lot de traces GPS
\item{} un processus de map matching pour une acquisition en temps réel
\item{} une interface graphique pour visualiser les données 3D (traces, réseau routier, fond de carte, \dots)
\end{itemize}
\subsection{Ressources}
\subsubsection{Temporelle}
Dans le cadre de la réalisation du projet, nous avons disposé de 180 heures réparties entre le 3 Octobre et le 6 Décembre, date de la soutenance et de la clôture du projet. Le projet a été développé sur 60 demies-journée de 3 heures.
\subsubsection{Humaine}
Notre équipe est composée de 7 individus, tous développeurs. Parmi le groupe, une personne a accepté le rôle de Scrum-Master (Romain Mazière) pour la cohésion de l'équipe, la mise en place des sprints et la gestion du code. Deux autres membres se sont proposés à endosser le rôle de Product-Owner (Camille Parisel et Rudolf Millet) dans le but d'écrire les Users Stories et les Tâches de chacun et d'assurer un lien entre le client (Benoît Costes) et le groupe de développeurs.
\subsubsection{Matériel}
Dès le début du projet, nous avons pu bénéficier d'ordinateurs équipés du système d'exploitation Debian. Notre choix d'IDE s'est tourné très tôt vers Qt Creator pour la mise en place d'IHM et vers la version 14 de Qgis qui nous propose le plugin Globe près à l'utilisation.

De plus, un cours d'une semaine de méthode agile nous a été enseigné, lequel nous a entrainé concrètement sur un petit projet. C'est suite à ce cours que les rôles ont été répartis et que la méthode Scrum a été appliquée.

Pour la gestion des tâches du projet, nous nous sommes servis du site web taiga.io qui propose un affichage clair et dynamique des sprints. De même pour la gestion du code, nous avons utilisé Git (GitHub et GitLab).

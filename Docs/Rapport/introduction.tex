\chapter{Introduction}
La localisation exacte des biens et des personnes est un enjeu fondamental du XXI\ieme siècle. L'une des applications sinon la plus connue est l'aide à la navigation. Avant de l'utiliser pour optimiser les trajets aussi bien commerciaux que personnels, il a fallu développer des algorithmes de traitement des données brutes. L'un d'entre eux est l'objet de ce projet. 
\section{Présentation du projet}
\subsection{Contexte}
Quelques dizaines de mètres correspondent à la précision moyenne du positionnement par satellite en mode cinématique (GPS de voiture, GPS de téléphone). Dans une application d'aide à la navigation, cet écart peut amener à une erreur de positionnement. Les algorithmes de calcul d'itinéraire peuvent alors fournir des résultats aberrants et ainsi perdre l'utilisateur. Pour éviter ce phénomène, il est possible d'appliquer un algorithme de traitement des données brutes qui va corriger ces dernières pour les positionner sur la route qui a le plus de chance d'avoir été empruntée. Les algorithmes de calculs d'itinéraires ont alors accès à des données qui correspondent davantage à la réalité de l'utilisateur.
\subsection{Objectif}
L'objectif de ce projet est de développer un outil de recalage de traces GPS (map matching) sur le réseau routier. Cela inclut :
\begin{itemize}
\item{} un processus pour traiter un lot de traces GPS
\item{} un processus appliqué à une acquisition en temps réel
\item{} une interface graphique pour visualiser les données 3D (traces, réseau routier, fond de carte, \dots)
\end{itemize}
\subsection{Ressources}
\subsection{\'Equipe}
\subsection{Outils utilisés}